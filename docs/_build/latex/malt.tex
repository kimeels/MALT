%% Generated by Sphinx.
\def\sphinxdocclass{report}
\documentclass[a4paper,10pt,english]{sphinxmanual}
\ifdefined\pdfpxdimen
   \let\sphinxpxdimen\pdfpxdimen\else\newdimen\sphinxpxdimen
\fi \sphinxpxdimen=.75bp\relax

\PassOptionsToPackage{warn}{textcomp}
\usepackage[utf8]{inputenc}
\ifdefined\DeclareUnicodeCharacter
% support both utf8 and utf8x syntaxes
  \ifdefined\DeclareUnicodeCharacterAsOptional
    \def\sphinxDUC#1{\DeclareUnicodeCharacter{"#1}}
  \else
    \let\sphinxDUC\DeclareUnicodeCharacter
  \fi
  \sphinxDUC{00A0}{\nobreakspace}
  \sphinxDUC{2500}{\sphinxunichar{2500}}
  \sphinxDUC{2502}{\sphinxunichar{2502}}
  \sphinxDUC{2514}{\sphinxunichar{2514}}
  \sphinxDUC{251C}{\sphinxunichar{251C}}
  \sphinxDUC{2572}{\textbackslash}
\fi
\usepackage{cmap}
\usepackage[T1]{fontenc}
\usepackage{amsmath,amssymb,amstext}
\usepackage{babel}



\usepackage{times}
\expandafter\ifx\csname T@LGR\endcsname\relax
\else
% LGR was declared as font encoding
  \substitutefont{LGR}{\rmdefault}{cmr}
  \substitutefont{LGR}{\sfdefault}{cmss}
  \substitutefont{LGR}{\ttdefault}{cmtt}
\fi
\expandafter\ifx\csname T@X2\endcsname\relax
  \expandafter\ifx\csname T@T2A\endcsname\relax
  \else
  % T2A was declared as font encoding
    \substitutefont{T2A}{\rmdefault}{cmr}
    \substitutefont{T2A}{\sfdefault}{cmss}
    \substitutefont{T2A}{\ttdefault}{cmtt}
  \fi
\else
% X2 was declared as font encoding
  \substitutefont{X2}{\rmdefault}{cmr}
  \substitutefont{X2}{\sfdefault}{cmss}
  \substitutefont{X2}{\ttdefault}{cmtt}
\fi


\usepackage[Bjarne]{fncychap}
\usepackage{sphinx}

\fvset{fontsize=\small}
\usepackage{geometry}

% Include hyperref last.
\usepackage{hyperref}
% Fix anchor placement for figures with captions.
\usepackage{hypcap}% it must be loaded after hyperref.
% Set up styles of URL: it should be placed after hyperref.
\urlstyle{same}
\addto\captionsenglish{\renewcommand{\contentsname}{Contents:}}

\usepackage{sphinxmessages}
\setcounter{tocdepth}{1}



\title{MALT}
\date{Sep 19, 2019}
\release{1.0}
\author{Kimeel}
\newcommand{\sphinxlogo}{\vbox{}}
\renewcommand{\releasename}{Release}
\makeindex
\begin{document}

\pagestyle{empty}
\sphinxmaketitle
\pagestyle{plain}
\sphinxtableofcontents
\pagestyle{normal}
\phantomsection\label{\detokenize{index::doc}}



\chapter{How to install MALT}
\label{\detokenize{install:how-to-install-malt}}\label{\detokenize{install::doc}}
First clone the git repo and install virtualenv:

\begin{sphinxVerbatim}[commandchars=\\\{\}]
\PYG{n}{git} \PYG{n}{clone} \PYG{n}{https}\PYG{p}{:}\PYG{o}{/}\PYG{o}{/}\PYG{n}{github}\PYG{o}{.}\PYG{n}{com}\PYG{o}{/}\PYG{n}{kimeels}\PYG{o}{/}\PYG{n}{MALT}\PYG{o}{.}\PYG{n}{git}

\PYG{n}{python3} \PYG{o}{\PYGZhy{}}\PYG{n}{m} \PYG{n}{pip} \PYG{n}{install} \PYG{o}{\PYGZhy{}}\PYG{o}{\PYGZhy{}}\PYG{n}{user} \PYG{n}{virtualenv}
\end{sphinxVerbatim}

Change directories into MALT and create a virtual environment:

\begin{sphinxVerbatim}[commandchars=\\\{\}]
\PYG{n}{cd} \PYG{n}{MALT}

\PYG{n}{python3} \PYG{o}{\PYGZhy{}}\PYG{n}{m} \PYG{n}{venv} \PYG{n}{malt\PYGZus{}env}
\end{sphinxVerbatim}

Start the virtual env and install the necessary packages using the requirements file:

\begin{sphinxVerbatim}[commandchars=\\\{\}]
\PYG{n}{source} \PYG{n}{malt\PYGZus{}env}\PYG{o}{/}\PYG{n+nb}{bin}\PYG{o}{/}\PYG{n}{activate}

\PYG{n}{pip3} \PYG{n}{install} \PYG{o}{\PYGZhy{}}\PYG{n}{r} \PYG{n}{requirements}\PYG{o}{.}\PYG{n}{txt}
\end{sphinxVerbatim}


\chapter{The MALT API reference}
\label{\detokenize{api:the-malt-api-reference}}\label{\detokenize{api::doc}}

\section{The Lightcurve class}
\label{\detokenize{api:the-lightcurve-class}}\index{Lightcurve (class in malt)@\spxentry{Lightcurve}\spxextra{class in malt}}

\begin{fulllineitems}
\phantomsection\label{\detokenize{api:malt.Lightcurve}}\pysiglinewithargsret{\sphinxbfcode{\sphinxupquote{class }}\sphinxcode{\sphinxupquote{malt.}}\sphinxbfcode{\sphinxupquote{Lightcurve}}}{\emph{filepath}, \emph{interpolate=False}, \emph{interp\_func=\textless{}function get\_gp\textgreater{}}, \emph{ini\_t='rand'}, \emph{obs\_time=0.3333333333333333}, \emph{sample\_size=100}, \emph{obj\_type=None}}{}~\index{extract\_features() (malt.Lightcurve method)@\spxentry{extract\_features()}\spxextra{malt.Lightcurve method}}

\begin{fulllineitems}
\phantomsection\label{\detokenize{api:malt.Lightcurve.extract_features}}\pysiglinewithargsret{\sphinxbfcode{\sphinxupquote{extract\_features}}}{\emph{feat\_ex\_method=\textless{}function get\_wavelet\_feature\textgreater{}}}{}
Extracts features from the given lightcurve with assigned
feature extraction method.
\begin{description}
\item[{self}] \leavevmode{[}Lightcurve object{]}
An instance of the Lightcurve class.

\item[{feat\_ex\_method: python function}] \leavevmode
Function to use for the feature extraction.

\end{description}

\end{fulllineitems}

\index{interpolate() (malt.Lightcurve method)@\spxentry{interpolate()}\spxextra{malt.Lightcurve method}}

\begin{fulllineitems}
\phantomsection\label{\detokenize{api:malt.Lightcurve.interpolate}}\pysiglinewithargsret{\sphinxbfcode{\sphinxupquote{interpolate}}}{\emph{interp\_func=\textless{}function get\_gp\textgreater{}}, \emph{ini\_t='rand'}, \emph{obs\_time=0.3333333333333333}, \emph{sample\_size=100}, \emph{aug\_num=1}}{}
Interpolates the given lightcurve with assigned interpolation function
\begin{description}
\item[{self: Lightcurve object}] \leavevmode
An instance of the Lightcurve class.

\item[{interp\_func: python function}] \leavevmode
A python function that takes in a lightcurve and interpolates it.

\item[{ini\_t: str or float}] \leavevmode
Initial time to start sampling.

\item[{obs\_time: float}] \leavevmode
The total length of the interpolated lightcurve.

\item[{sample\_size: int}] \leavevmode
Number of data points in interpolated lightcurve.

\item[{aug\_num: int}] \leavevmode
Number of lightcurves to augment to.

\end{description}

\end{fulllineitems}

\index{loadfile() (malt.Lightcurve method)@\spxentry{loadfile()}\spxextra{malt.Lightcurve method}}

\begin{fulllineitems}
\phantomsection\label{\detokenize{api:malt.Lightcurve.loadfile}}\pysiglinewithargsret{\sphinxbfcode{\sphinxupquote{loadfile}}}{\emph{filename}}{}
Loads file to extract time, flux, flux\_err  ra\_dec and class

filename: path to dataset

\end{fulllineitems}


\end{fulllineitems}



\section{The Dataset class}
\label{\detokenize{api:the-dataset-class}}\index{Dataset (class in malt)@\spxentry{Dataset}\spxextra{class in malt}}

\begin{fulllineitems}
\phantomsection\label{\detokenize{api:malt.Dataset}}\pysiglinewithargsret{\sphinxbfcode{\sphinxupquote{class }}\sphinxcode{\sphinxupquote{malt.}}\sphinxbfcode{\sphinxupquote{Dataset}}}{\emph{configFile='', feat\_ex\_method=\textless{}function get\_wavelet\_feature\textgreater{}, interpolate=True, interp\_func=\textless{}function get\_gp\textgreater{}, ini\_t='rand', obs\_time=0.3333333333333333, sample\_size=100, aug\_num=1, ml\_method=\textless{}class 'malt.machine\_learning.RFclassifier'\textgreater{}, hyperparams=\{'criterion': {[}'gini', 'entropy'{]}, 'n\_estimators': array({[}70, 71, 72, 73, 74, 75, 76, 77, 78, 79, 80, 81, 82, 83, 84, 85, 86,        87, 88, 89{]})\}, n\_jobs=-1, pca=True, n\_components=20}}{}~\index{add() (malt.Dataset method)@\spxentry{add()}\spxextra{malt.Dataset method}}

\begin{fulllineitems}
\phantomsection\label{\detokenize{api:malt.Dataset.add}}\pysiglinewithargsret{\sphinxbfcode{\sphinxupquote{add}}}{\emph{new\_lightcurve}}{}~\begin{quote}

Adds new lightcurve to the Dataset then retrains Dataset.
\end{quote}
\begin{description}
\item[{self: Dataset object}] \leavevmode
An instance of the Dataset class containing instances of the
Lightcurve class.

\item[{lightcurve}] \leavevmode{[}Lightcurve object{]}
Lightcurve object to add to dataset.

\end{description}

\end{fulllineitems}

\index{extract\_features() (malt.Dataset method)@\spxentry{extract\_features()}\spxextra{malt.Dataset method}}

\begin{fulllineitems}
\phantomsection\label{\detokenize{api:malt.Dataset.extract_features}}\pysiglinewithargsret{\sphinxbfcode{\sphinxupquote{extract\_features}}}{}{}
Extracts features from all the lightcurves in the given dataset with
assigned feature extraction method.
\begin{description}
\item[{self}] \leavevmode{[}Dataset object{]}
An instance of the Dataset class containing instances of the
Lightcurve class.

\end{description}

\end{fulllineitems}

\index{get\_pca() (malt.Dataset method)@\spxentry{get\_pca()}\spxextra{malt.Dataset method}}

\begin{fulllineitems}
\phantomsection\label{\detokenize{api:malt.Dataset.get_pca}}\pysiglinewithargsret{\sphinxbfcode{\sphinxupquote{get\_pca}}}{}{}
Performs PCA decomposition of a feature array X.
\begin{description}
\item[{self: Dataset object}] \leavevmode
An instance of the Dataset class containing instances of the
Lightcurve class.

\end{description}

\end{fulllineitems}

\index{interpolate() (malt.Dataset method)@\spxentry{interpolate()}\spxextra{malt.Dataset method}}

\begin{fulllineitems}
\phantomsection\label{\detokenize{api:malt.Dataset.interpolate}}\pysiglinewithargsret{\sphinxbfcode{\sphinxupquote{interpolate}}}{}{}
Interpolates all the lightcurves in the given dataset with
assigned interpolation function.
\begin{description}
\item[{self}] \leavevmode{[}Dataset object{]}
An instance of the Dataset class containing instances of the
Lightcurve class.

\end{description}

\end{fulllineitems}

\index{load\_from\_save() (malt.Dataset class method)@\spxentry{load\_from\_save()}\spxextra{malt.Dataset class method}}

\begin{fulllineitems}
\phantomsection\label{\detokenize{api:malt.Dataset.load_from_save}}\pysiglinewithargsret{\sphinxbfcode{\sphinxupquote{classmethod }}\sphinxbfcode{\sphinxupquote{load\_from\_save}}}{\emph{filename}}{}~\begin{quote}

Returns a saved Dataset instance using pickle
\end{quote}
\begin{description}
\item[{self: Dataset object}] \leavevmode
An instance of the Dataset class containing instances of the
Lightcurve class.

\item[{filename: str}] \leavevmode
filename under which the Dataset instance was saved.

\end{description}

\end{fulllineitems}

\index{populate() (malt.Dataset method)@\spxentry{populate()}\spxextra{malt.Dataset method}}

\begin{fulllineitems}
\phantomsection\label{\detokenize{api:malt.Dataset.populate}}\pysiglinewithargsret{\sphinxbfcode{\sphinxupquote{populate}}}{\emph{filepaths}}{}
Initialises an instance of the Dataset class.
\begin{description}
\item[{self}] \leavevmode{[}Database object{]}
An instance of the Database class.

\item[{filepaths: list}] \leavevmode
List containing the paths to the data files.

\end{description}

\end{fulllineitems}

\index{predict() (malt.Dataset method)@\spxentry{predict()}\spxextra{malt.Dataset method}}

\begin{fulllineitems}
\phantomsection\label{\detokenize{api:malt.Dataset.predict}}\pysiglinewithargsret{\sphinxbfcode{\sphinxupquote{predict}}}{\emph{lightcurve}, \emph{show\_prob=False}}{}~\begin{quote}

Predicts the type of given lightcurve object using classifier trained
on Dataset.
\end{quote}
\begin{description}
\item[{self}] \leavevmode{[}Dataset object{]}
An instance of the Dataset class containing instances of the
Lightcurve class.

\item[{lightcurve}] \leavevmode{[}Lightcurve object{]}
Lightcurve object for which to predict

\item[{show\_prob}] \leavevmode{[}boolean.{]}
If True will print full output from predict\_proba()

\end{description}

\end{fulllineitems}

\index{project\_pca() (malt.Dataset method)@\spxentry{project\_pca()}\spxextra{malt.Dataset method}}

\begin{fulllineitems}
\phantomsection\label{\detokenize{api:malt.Dataset.project_pca}}\pysiglinewithargsret{\sphinxbfcode{\sphinxupquote{project\_pca}}}{\emph{lightcurve=None}}{}
Projects self.features onto  calculated PCA axis from self.pca
\begin{description}
\item[{self: Dataset object}] \leavevmode
An instance of the Dataset class containing instances of the
Lightcurve class.

\end{description}

\end{fulllineitems}

\index{run\_diagnostic() (malt.Dataset method)@\spxentry{run\_diagnostic()}\spxextra{malt.Dataset method}}

\begin{fulllineitems}
\phantomsection\label{\detokenize{api:malt.Dataset.run_diagnostic}}\pysiglinewithargsret{\sphinxbfcode{\sphinxupquote{run\_diagnostic}}}{}{}
Runs the Diagnostic test which trains n classifiers on different subsets
of the Dataset to test how well it can classify objects.
\begin{description}
\item[{self: Dataset object}] \leavevmode
An instance of the Dataset class containing instances of the
Lightcurve class.

\end{description}

\end{fulllineitems}

\index{save() (malt.Dataset method)@\spxentry{save()}\spxextra{malt.Dataset method}}

\begin{fulllineitems}
\phantomsection\label{\detokenize{api:malt.Dataset.save}}\pysiglinewithargsret{\sphinxbfcode{\sphinxupquote{save}}}{\emph{filename='saved\_dataset'}}{}~\begin{quote}

Saves a Dataset instance using a pickle dump
\end{quote}
\begin{description}
\item[{self: Dataset object}] \leavevmode
An instance of the Dataset class containing instances of the
Lightcurve class.

\item[{filename: str}] \leavevmode
filename under which to store the Dataset instance

\end{description}

\end{fulllineitems}

\index{train() (malt.Dataset method)@\spxentry{train()}\spxextra{malt.Dataset method}}

\begin{fulllineitems}
\phantomsection\label{\detokenize{api:malt.Dataset.train}}\pysiglinewithargsret{\sphinxbfcode{\sphinxupquote{train}}}{\emph{verbose=1}}{}~\begin{quote}

Trains a ML algorithm on the Dataset with the parameters specified on
initialisation.
\end{quote}
\begin{description}
\item[{self}] \leavevmode{[}Dataset object{]}
An instance of the Dataset class containing instances of the
Lightcurve class.

\end{description}

verbose : How much information to print out.

\end{fulllineitems}

\index{types() (malt.Dataset method)@\spxentry{types()}\spxextra{malt.Dataset method}}

\begin{fulllineitems}
\phantomsection\label{\detokenize{api:malt.Dataset.types}}\pysiglinewithargsret{\sphinxbfcode{\sphinxupquote{types}}}{\emph{show\_aug\_num=False}}{}
Prints out the counts of each object type stored in the dataset.
\begin{description}
\item[{self: Dataset object}] \leavevmode
An instance of the Dataset class containing instances of the
Lightcurve class.

\item[{show\_aug\_num: boolean}] \leavevmode
Use augmented lightcurve when counting type numbers.

\end{description}

\end{fulllineitems}


\end{fulllineitems}



\section{MALT interpolator}
\label{\detokenize{api:module-malt.interpolator}}\label{\detokenize{api:malt-interpolator}}\index{malt.interpolator (module)@\spxentry{malt.interpolator}\spxextra{module}}\index{get\_gp() (in module malt.interpolator)@\spxentry{get\_gp()}\spxextra{in module malt.interpolator}}

\begin{fulllineitems}
\phantomsection\label{\detokenize{api:malt.interpolator.get_gp}}\pysiglinewithargsret{\sphinxcode{\sphinxupquote{malt.interpolator.}}\sphinxbfcode{\sphinxupquote{get\_gp}}}{\emph{lightcurve}, \emph{t0}, \emph{obs\_time}, \emph{sample\_size}, \emph{aug\_num}}{}
Returns a Gaussian Process (george) object marginalised on the data
in file.
\begin{description}
\item[{lightcurve: Lightcurve object}] \leavevmode
An instance of the Lightcurve class.

\item[{t0: float}] \leavevmode
Initial time to start sampling.

\item[{obs\_time: float}] \leavevmode
The total length of the interpolated lightcurve.

\item[{sample\_size: int}] \leavevmode
Number of data points in interpolated lightcurve.

\end{description}

\end{fulllineitems}



\section{MALT feature extraction}
\label{\detokenize{api:module-malt.feature_extraction}}\label{\detokenize{api:malt-feature-extraction}}\index{malt.feature\_extraction (module)@\spxentry{malt.feature\_extraction}\spxextra{module}}\index{get\_wavelet\_feature() (in module malt.feature\_extraction)@\spxentry{get\_wavelet\_feature()}\spxextra{in module malt.feature\_extraction}}

\begin{fulllineitems}
\phantomsection\label{\detokenize{api:malt.feature_extraction.get_wavelet_feature}}\pysiglinewithargsret{\sphinxcode{\sphinxupquote{malt.feature\_extraction.}}\sphinxbfcode{\sphinxupquote{get\_wavelet\_feature}}}{\emph{lightcurve}}{}
Returns wavelet coefficients for a given lightcurve object.
\begin{description}
\item[{lightcurve}] \leavevmode{[}Lightcurve object{]}
An instance of the Lightcurve class

\end{description}

\end{fulllineitems}



\section{MALT machine learning}
\label{\detokenize{api:module-malt.machine_learning}}\label{\detokenize{api:malt-machine-learning}}\index{malt.machine\_learning (module)@\spxentry{malt.machine\_learning}\spxextra{module}}\index{RFclassifier (class in malt.machine\_learning)@\spxentry{RFclassifier}\spxextra{class in malt.machine\_learning}}

\begin{fulllineitems}
\phantomsection\label{\detokenize{api:malt.machine_learning.RFclassifier}}\pysiglinewithargsret{\sphinxbfcode{\sphinxupquote{class }}\sphinxcode{\sphinxupquote{malt.machine\_learning.}}\sphinxbfcode{\sphinxupquote{RFclassifier}}}{\emph{n\_estimators='warn'}, \emph{criterion='gini'}}{}
\end{fulllineitems}



\renewcommand{\indexname}{Python Module Index}
\begin{sphinxtheindex}
\let\bigletter\sphinxstyleindexlettergroup
\bigletter{m}
\item\relax\sphinxstyleindexentry{malt.feature\_extraction}\sphinxstyleindexpageref{api:\detokenize{module-malt.feature_extraction}}
\item\relax\sphinxstyleindexentry{malt.interpolator}\sphinxstyleindexpageref{api:\detokenize{module-malt.interpolator}}
\item\relax\sphinxstyleindexentry{malt.machine\_learning}\sphinxstyleindexpageref{api:\detokenize{module-malt.machine_learning}}
\end{sphinxtheindex}

\renewcommand{\indexname}{Index}
\printindex
\end{document}